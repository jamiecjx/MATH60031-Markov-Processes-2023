\documentclass{article}
\usepackage{graphicx} % Required for inserting images
\usepackage{amsmath} % useful
\usepackage{amsfonts} % fonts like \mathbb{R} among others
\usepackage{amssymb}
\usepackage{amsthm} % used for writing lemmas/theorems/etc
\usepackage{tikz} % for visual tools
\usepackage{authblk}
\usepackage[style=numeric]{biblate  x}
\addbibresource{references.bib}
\usepackage{enumitem}
\numberwithin{equation}{section}
\usepackage[margin=1in]{geometry}

\newtheorem{theorem}{Theorem}[section]
\newtheorem{corollary}{Corollary}[theorem]
\newtheorem{lemma}[theorem]{Lemma}
\newtheorem{proposition}[theorem]{Proposition}

\theoremstyle{definition}
\newtheorem{definition}{Definition}[section]
\newtheorem{remark}{Remark}[theorem]


\title{MATH60031 Markov Processes}
\author{Lectured by Dr Ajay Chandra\\
Autumn 2023\\
}
\date{}


\newcommand{\R}{\mathbb{R}}
\newcommand{\C}{\mathbb{C}}
\newcommand{\Q}{\mathbb{Q}}
\newcommand{\pr}{\mathbb{P}}
\newcommand{\E}{\mathbb{E}}
\newcommand{\dm}{\mathrm{d}}
\newcommand{\borelx}{\mc{B}(\mc{X})}

\newcommand{\mc}[1]{\mathcal{#1}}
\newcommand{\pspace}{(\Omega, \mathcal{F}, \mathbb{P})}

\newcommand{\sm}{\setminus}

\newcommand{\ie}{\textit{i}.\textit{e}.}
\newcommand{\eg}{\textit{e}.\textit{g}.}

\begin{document}

\maketitle


\section{Basics}
\begin{definition}
    A probability space is a triple $\pspace$ consisting of a state space $\Omega$, a $\sigma$-algebra $\mc{F}$ and a probability measure $\pr$.
    A state space $\mc{X}$ is assumed to be a complete seperable metric space. Let $\borelx$ be the Borel $\sigma$-algebra.
\end{definition}

\begin{definition}
    Given the above setup, a Stochastic Process is a collection of random variables/measurable functions from $(\Omega,\mc{F})$ to $(\mc{X}, \borelx)$. The indexing of the random variables is a totally ordered set usually.
\end{definition}

\subsection{Information and Filtrations}
\begin{definition}
    The information on the stochastic process at time $n$ is the $\sigma$-algebra of all possible events at this time.

    \begin{equation}
        \sigma(X_n) = \sigma
        \left(
            \left\{
                X_n^{-1}(A): A\in\borelx
            \right\}
        \right)
    \end{equation}
    This can be thought of as all the information we can find about $\Omega$ given $X_n$. Think about what would happen if $X_n$ was constant: the $\sigma$ algebra is trivial and you have no information about it.
\end{definition}
\begin{definition}
    $\sigma(X_0, X_1, \ldots ,X_n)$ is the smallest $\sigma$-algebra such that each random variable is measurable. This can be thought of as the information up to time $n$.
\end{definition}
\begin{definition}
    A family of $\sigma$-algebras $\left\{\mc{F}_n\right\}$ satisfying $\mc{F}_n \subset \mc{F}_m$ whenever $n<m$ is a filtration. This makes sure that we can't see extra information in the future when we haven't reached it yet.
    \begin{itemize}
        \item A stochastic process $X_n$ is said to be adapted to a filtration $\mc{F}_n$ if for all $n$, $X_n$ is measurable with respect to $\mc{F}_n$.
        \item The filtration $\mc{F}_n^X =  \sigma(X_0, X_1, \ldots ,X_n)$ is the smallest filtration, and is called the natural filtration.
    \end{itemize}
\end{definition}
\end{document}
